\documentclass{article}
\usepackage{verbatim}
\usepackage{graphicx}

\begin{document}
METODOS COMPUTACIONALES HW5\\

El siguiente documento muestra los resultados de la solucion a los dos problemas presentados en la Tarea 5. Em primer lugar, se muestran las graficas resultantes del problema de biofisica, y luego las del circuito:\\
BIOFISICA, PRIMER MOLECULA:\\

\begin{center}
\includegraphics[scale=0.60]{XH.pdf}
\end{center}

\begin{center}
\includegraphics[scale=0.60]{YH.pdf}
\end{center}

\begin{center}
\includegraphics[scale=0.60]{FitCircle.pdf}
\end{center}
Como se puede ver en la anterior secuencia de graficas, los puntos no se mueven bastante de su posicion final, lo que habla de la facilidad y exitosa convergencia del metodo. Se puede apreciar como el circulo mas grande esta en el lado izquierdo.\\
\\

BIOFISICA, SEGUNDA MOLECULA:\\

\begin{center}
\includegraphics[scale=0.60]{X1H.pdf}
\end{center}

\begin{center}
\includegraphics[scale=0.60]{Y1H.pdf}
\end{center}

\begin{center}
\includegraphics[scale=0.60]{FitCircle1.pdf}
\end{center}
Como se puede ver en la anterior secuencia de graficas, los puntos no se mueven bastante de su posicion final, al igual que la anterior, lo que habla de la facilidad y exitosa convergencia del metodo. Se puede apreciar como el circulo mas grande esta en el lado izquierdo derecho, contrario al anterior, pero con un radio menor al primero.\\
Ahora, se presentan los resultados del ejercicio del circuito:
\\
\begin{center}
\includegraphics[scale=0.60]{CH.pdf}
\end{center}

\begin{center}
\includegraphics[scale=0.60]{RH.pdf}
\end{center}

\begin{center}
\includegraphics[scale=0.60]{CL.pdf}
\end{center}

\begin{center}
\includegraphics[scale=0.60]{RL.pdf}
\end{center}

\begin{center}
\includegraphics[scale=0.60]{Fit.pdf}
\end{center}

Como se puede observar, los histogramas y graficas contra el likelihood muestran que la convergencia no fue dificil, dado que estan centrados sobre el resultado final. Con la ultima grafica se puede evidenciar que fue bien desarrollado el metodo, obteniendo parametros adecuados y una buena estimacion de la funcion.

\end{document}
