\documentclass{article}
\usepackage{verbatim}
\usepackage{graphicx}

\begin{document}
METODOS COMPUTACIONALES HW4\\

El siguiente documento muestra los resultados de la solucion numerica de la ecuacion de difusion en dos dimensiones para dos casos iniciales diferentes en sus distintas condiciones de frontera. La primera condicion establece un rectángulo dentro de la placa más caliente que el resto. Las siguientes son las gráficas para dicho caso, en cada una de las condiciones de frontera, para diferentes tiempos:\\

\begin{center}
\includegraphics[scale=0.60]{CASO1t0sfronteraCerrada.png}
\end{center}

\begin{center}
\includegraphics[scale=0.60]{CASO1t100sfronteraCerrada.png}
\end{center}

\begin{center}
\includegraphics[scale=0.60]{CASO1t2500sfronteraCerrada.png}
\end{center}
Como se puede ver en la anterior secuencia, dado que la frontera es cerrada, tiende a la temperatura de esta, la cual es 50 centigrados.\\
\\

\begin{center}
\includegraphics[scale=0.60]{CASO1t0sfronteraPeriodica.png}
\end{center}

\begin{center}
\includegraphics[scale=0.60]{CASO1t100sfronteraPeriodica.png}
\end{center}

\begin{center}
\includegraphics[scale=0.60]{CASO1t2500sfronteraPeriodica.png}
\end{center}
Como se puede ver en la anterior secuencia, dado que la frontera es Periodica, tiende a la temperatura interna de esta, la cual es 50 centigrados, dado que se entiende como un flujo continuo. \\
\\

\begin{center}
\includegraphics[scale=0.60]{CASO1t0sfronteraAbierta.png}
\end{center}

\begin{center}
\includegraphics[scale=0.60]{CASO1t100sfronteraAbierta.png}
\end{center}

\begin{center}
\includegraphics[scale=0.60]{CASO1t2500sfronteraAbierta.png}
\end{center}
Como se puede ver en la anterior secuencia, dado que la frontera es Abierta, tiende a la temperatura que influencia el rectangulo, mayor a 50 levemente mas cerca de este y menor donde es mas lejos.\\
\\
Por otro lado, el caso 2 tiene el mismo rectángulo, solo que este siempre permanece con la misma temperatura de 100 grados en todo el intervalo de tiempo:\\

\begin{center}
\includegraphics[scale=0.60]{CASO2t0sfronteraCerrada.png}
\end{center}

\begin{center}
\includegraphics[scale=0.60]{CASO2t100sfronteraCerrada.png}
\end{center}

\begin{center}
\includegraphics[scale=0.60]{CASO2t2500sfronteraCerrada.png}
\end{center}
Como se puede ver en la anterior secuencia, dado que la frontera es cerrada, tiende a la temperatura de esta, la cual es 50 centigrados, mientras que se calienta de forma sistematica los alrededores del rectangulo.\\
\\

\begin{center}
\includegraphics[scale=0.60]{CASO2t0sfronteraPeriodica.png}
\end{center}

\begin{center}
\includegraphics[scale=0.60]{CASO2t100sfronteraPeriodica.png}
\end{center}

\begin{center}
\includegraphics[scale=0.60]{CASO2t2500sfronteraPeriodica.png}
\end{center}
Como se puede ver en la anterior secuencia, dado que la frontera es Periodica, tiende a la temperatura interna de esta, la cual es 50 centigrados, dado que se entiende como un flujo continuo sobre el rectangulo. \\
\\

\begin{center}
\includegraphics[scale=0.60]{CASO2t0sfronteraAbierta.png}
\end{center}

\begin{center}
\includegraphics[scale=0.60]{CASO2t100sfronteraAbierta.png}
\end{center}

\begin{center}
\includegraphics[scale=0.60]{CASO2t2500sfronteraAbierta.png}
\end{center}
Como se puede ver en la anterior secuencia, dado que la frontera es Abierta, tiende a la temperatura que influencia el rectangulo, mayor a 50 levemente mas cerca de este y menor donde es mas lejos, por lo que tiende a calentarse hasta temperaturas bastante cercanas a 100 los alrededores del rectangulo.\\
\\
Finalmente, se pueden observar los promedios de temperatura para cada uno de los casos y fronteras:\\

\begin{center}
\includegraphics[scale=0.60]{Caso1Mean.png}
\end{center}
Dada la libertad de la frontera abierta, el calor no se estabiliza de forma rapida en una temperatura central sino que disminuye constantemente y mas rapido, a diferencia de los otros dos que tienen comportamientos muy parecidos.\\
\\
De la misma forma, este es el resultado del caso 2:
\begin{center}
\includegraphics[scale=0.60]{Caso2Mean.png}
\end{center}
Aqui, se tiene el mismo concepto, solo que dadas las condiciones iniciales tenderá a calentarse por la fuente constante que existe. De igual forma, el proceso es mas rapido para la frontera abierta que para las otras.

\end{document}
